\documentclass[a2,portrait]{a0poster}


% PACKAGES
% ========
% Do not indent paragraphs and insert blank space between paragraphs.
\usepackage{parskip}

% To arrange content in multiple columns.
\usepackage{multicol}

% To define and use colors.
\usepackage[svgnames]{xcolor}

% Use Times font for the text.
\renewcommand{\rmdefault}{ptm}

% Commands like \textasciigrave, \textquotesingle, etc. are required by
% listings when \lstset{upquote=true} is used. They are defined in
% textcomp.sty.
\usepackage{textcomp}

% To enter syntax-highlighted code blocks.
\usepackage{listings}

% To color links.
\usepackage[colorlinks=true,urlcolor=DarkBlue]{hyperref}

% To include images.
\usepackage{graphicx}


% FORMATTING
% ==========
% Use current text font for links (override default teletype font).
\urlstyle{same}


% LAYOUT
% ======
% Increase column separation.
\setlength{\columnsep}{30mm}

% Increase text height to make the whole code fit.
\addtolength{\topmargin}{-12mm}
\addtolength{\textheight}{15mm}


% CODE LISTINGS
% =============
% Colors for syntax highlighting.
\definecolor{codecolor}{HTML}{000060} % dark blue
\definecolor{commentcolor}{HTML}{586e75} % bluish gray
\definecolor{keywordcolor}{HTML}{0000f0} % blue
\definecolor{stringcolor}{HTML}{600000} % dark red
\definecolor{placeholdercolor}{HTML}{006000} % dark green

% Base style for code listing.
\lstset{
    % Use small teletype font (override default roman font).
    basicstyle=\ttfamily\color{codecolor},
    % Use natural width of characters and do not mess with alignment.
    columns=fullflexible,
    % Do not drop consecutive spaces.
    keepspaces=true,
    % Use straight quotes instead of curved quotes.
    upquote=true,
    % Do not show spaces in strings as visible spaces.
    showstringspaces=false,
    % Set colors for syntax highlighting.
    commentstyle=\color{commentcolor},
    keywordstyle=\color{keywordcolor},
    stringstyle=\color{stringcolor},
    numberstyle=\color{gray},
    % Show line numbers.
    numbers=left,
    % Allow line numbers to continue from one listing to another.
    firstnumber=last,
    % Show empty lines at the end of listings.
    showlines=true,
}

% Add Python keywords unrecognized by listings for highlighting.
\lstnewenvironment{pythoncode}{
    \lstset{
        language=python,
        morekeywords={yield, True, False},
    }
}{}

% Display inline code.
\newcommand{\inlinecode}[1]{%
    \lstinline{#1}%
}


% IMAGES
% ======
% Image files search path.
\graphicspath{{../2018-09-17-pycon-uk-makesite-talk/img/}}


% DOCUMENT
% ========
\begin{document}
% Header
\begin{minipage}[b]{0.40\linewidth}
    % Title
    \normalsize
    \color{Navy}
    \textbf{Source Code (\texttt{makesite.py})}

    \smallskip

    % Author
    \small
    \color{black}
    \textbf{Sunaina Pai}

    \smallskip

    % Conference
    PyCon UK 2018, Cardiff, UK
\end{minipage}
%
\hfill
%
\begin{minipage}[b]{0.40\linewidth}
    \footnotesize
    \color{DarkSlateGray}
    \textbf{GitHub URL:} \\
    \url{https://github.com/sunainapai/makesite}

    \vspace{1mm}

    \textbf{Short URL:} \\
    \url{https://git.io/makesite}
\end{minipage}
%
\newcommand{\qrsize}{28mm}
\begin{minipage}[b]{\qrsize}
    \href{https://github.com/sunainapai/makesite}{%
        \includegraphics[width=\qrsize]{makesite-qr.png}%
    }
\end{minipage}

\bigskip

\tiny

% Two columns for content.
\begin{multicols*}{2}
\begin{pythoncode}
"""Make static website/blog with Python."""


import os
import shutil
import re
import glob
import sys
import json
import datetime


def fread(filename):
    """Read file and close the file."""
    with open(filename, 'r') as f:
        return f.read()


def fwrite(filename, text):
    """Write content to file and close the file."""
    basedir = os.path.dirname(filename)
    if not os.path.isdir(basedir):
        os.makedirs(basedir)

    with open(filename, 'w') as f:
        f.write(text)


def log(msg, *args):
    """Log message with specified arguments."""
    sys.stderr.write(msg.format(*args) + '\n')


def truncate(text, words=25):
    """Remove tags and truncate text to the specified number of words."""
    return ' '.join(re.sub('(?s)<.*?>', ' ', text).split()[:words])


def read_headers(text):
    """Parse headers in text and yield (key, value, end-index) tuples."""
    for match in re.finditer('\s*<!--\s*(.+?)\s*:\s*(.+?)\s*-->\s*|.+', text):
        if not match.group(1):
            break
        yield match.group(1), match.group(2), match.end()


def rfc_2822_format(date_str):
    """Convert yyyy-mm-dd date string to RFC 2822 format date string."""
    d = datetime.datetime.strptime(date_str, '%Y-%m-%d')
    return d.strftime('%a, %d %b %Y %H:%M:%S +0000')


def read_content(filename):
    """Read content and metadata from file into a dictionary."""
    # Read file content.
    text = fread(filename)

    # Read metadata and save it in a dictionary.
    date_slug = os.path.basename(filename).split('.')[0]
    match = re.search('^(?:(\d\d\d\d-\d\d-\d\d)-)?(.+)$', date_slug)
    content = {
        'date': match.group(1) or '1970-01-01',
        'slug': match.group(2),
    }

    # Read headers.
    end = 0
    for key, val, end in read_headers(text):
        content[key] = val

    # Separate content from headers.
    text = text[end:]

    # Convert Markdown content to HTML.
    if filename.endswith(('.md', '.mkd', '.mkdn', '.mdown', '.markdown')):
        try:
            if _test == 'ImportError':
                raise ImportError('Error forced by test')
            import CommonMark
            text = CommonMark.commonmark(text)
        except ImportError as e:
            log('WARNING: Cannot render Markdown in {}: {}', filename, str(e))

    # Update the dictionary with content and RFC 2822 date.
    content.update({
        'content': text,
        'rfc_2822_date': rfc_2822_format(content['date'])
    })

    return content


def render(template, **params):
    """Replace placeholders in template with values from params."""
    return re.sub(r'{{\s*([^}\s]+)\s*}}',
                  lambda match: str(params.get(match.group(1), match.group(0))),
                  template)


\end{pythoncode}

\columnbreak

\begin{pythoncode}
def make_pages(src, dst, layout, **params):
    """Generate pages from page content."""
    items = []

    for src_path in glob.glob(src):
        content = read_content(src_path)

        page_params = dict(params, **content) 

        # Populate placeholders in content if content-rendering is enabled.
        if page_params.get('render') == 'yes':
            rendered_content = render(page_params['content'], **page_params)
            page_params['content'] = rendered_content
            content['content'] = rendered_content

        items.append(content)

        dst_path = render(dst, **page_params)
        output = render(layout, **page_params)

        log('Rendering {} => {} ...', src_path, dst_path)
        fwrite(dst_path, output)

    return sorted(items, key=lambda x: x['date'], reverse=True)


def make_list(posts, dst, list_layout, item_layout, **params):
    """Generate list page for a blog."""
    items = []
    for post in posts:
        item_params = dict(params, **post)
        item_params['summary'] = truncate(post['content'])
        item = render(item_layout, **item_params)
        items.append(item)

    params['content'] = ''.join(items)
    dst_path = render(dst, **params)
    output = render(list_layout, **params)

    log('Rendering list => {} ...', dst_path)
    fwrite(dst_path, output)


def main():
    # Create a new _site directory from scratch.
    if os.path.isdir('_site'):
        shutil.rmtree('_site')
    shutil.copytree('static', '_site')

    # Default parameters.
    params = {
        'base_path': '',
        'subtitle': 'Lorem Ipsum',
        'author': 'Admin',
        'site_url': 'http://localhost:8000',
        'current_year': datetime.datetime.now().year
    }

    # If params.json exists, load it.
    if os.path.isfile('params.json'):
        params.update(json.loads(fread('params.json')))

    # Load layouts.
    page_layout = fread('layout/page.html')
    post_layout = fread('layout/post.html')
    list_layout = fread('layout/list.html')
    item_layout = fread('layout/item.html')
    feed_xml = fread('layout/feed.xml')
    item_xml = fread('layout/item.xml')

    # Combine layouts to form final layouts.
    post_layout = render(page_layout, content=post_layout)
    list_layout = render(page_layout, content=list_layout)

    # Create site pages.
    make_pages('content/_index.html', '_site/index.html',
               page_layout, **params)
    make_pages('content/[!_]*.html', '_site/{{ slug }}/index.html',
               page_layout, **params)

    # Create blogs.
    blog_posts = make_pages('content/blog/*.md',
                            '_site/blog/{{ slug }}/index.html',
                            post_layout, blog='blog', **params)
    news_posts = make_pages('content/news/*.html',
                            '_site/news/{{ slug }}/index.html',
                            post_layout, blog='news', **params)

    # Create blog list pages.
    make_list(blog_posts, '_site/blog/index.html',
              list_layout, item_layout, blog='blog', title='Blog', **params)
    make_list(news_posts, '_site/news/index.html',
              list_layout, item_layout, blog='news', title='News', **params)

    # Create RSS feeds.
    make_list(blog_posts, '_site/blog/rss.xml',
              feed_xml, item_xml, blog='blog', title='Blog', **params)
    make_list(news_posts, '_site/news/rss.xml',
              feed_xml, item_xml, blog='news', title='News', **params)


# Test parameter to be set temporarily by unit tests.
_test = None


if __name__ == '__main__':
    main()
\end{pythoncode} 
\end{multicols*}

\end{document}
